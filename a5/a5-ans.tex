% You should title the file with a .tex extension (hw1.tex, for example)
%sudo apt-get install texstudio texlive-latex-extra

\documentclass[a4paper, 11pt]{article}
\usepackage{fancyvrb}
\usepackage{verbatim}
\usepackage{amsmath}
\usepackage{amssymb}
\usepackage{fancyhdr}
\usepackage{graphicx}

\usepackage[margin=1in]{geometry}
\usepackage{tikz}
\usetikzlibrary{automata,positioning,arrows}

\usepackage[margin=1in]{geometry}
\usepackage{listings}
\usepackage{color}

\definecolor{dkgreen}{rgb}{0,0.6,0}
\definecolor{gray}{rgb}{0.5,0.5,0.5}
\definecolor{mauve}{rgb}{0.58,0,0.82}

\lstset{frame=tb,
	language=Java,
	aboveskip=3mm,
	belowskip=3mm,
	showstringspaces=false,
	columns=flexible,
	basicstyle={\small\ttfamily},
	numbers=none,
	numberstyle=\tiny\color{gray},
	keywordstyle=\color{blue},
	commentstyle=\color{dkgreen},
	stringstyle=\color{mauve},
	breaklines=true,
	breakatwhitespace=true,
	tabsize=3
}

\newcommand{\question}[2] {\vspace{.25in} \hrule\vspace{0.5em}
	\noindent{\bf #1: #2} \vspace{0.5em}
	\hrule \vspace{.10in}}
\renewcommand{\part}[1] {\vspace{.10in} {\bf (#1)}}

\newcommand{\myname}{Possawat Sanorkam}
\newcommand{\myemail}{possawat2017@hotmail.com}
\newcommand{\myhwnum}{5}
\newcommand\tab[1][1cm]{\hspace*{#1}}

\setlength{\parindent}{0pt}
\setlength{\parskip}{5pt plus 1pt}

\pagestyle{fancyplain}
\lhead{\fancyplain{}{\textbf{HW\myhwnum}}}      % Note the different brackets!
\rhead{\fancyplain{}{\myname\\ \myemail}}
\chead{\fancyplain{}{ICCS310}}

\begin{document}
	
	\medskip                        % Skip a "medium" amount of space
	% (latex determines what medium is)
	% Also try: \bigskip, \littleskip
	
	\thispagestyle{plain}
	\begin{center}                  % Center the following lines
		{\Large ICCS310: Assignment \myhwnum} \\
		\myname \\
		\myemail \\
		\today \\
	\end{center}
	
	\question{1}{Reject TM} %don't delete yet:(}
	$$ \textsf{REJECT}_{\textsf{TM}} = \{\langle M , x \rangle | \text{ M is a TM that rejects input} \} $$
	
	Show directly (i.e., without resorting to reduction) that $\textsf{REJECT}_{\textsf{TM}}$ is undecidable.
	
	{\em Proof}: 
		
	\question{2}{Accept vs. Reject} %don't delete yet:(}
	
	$$ \textsf{REJECT}_{\textsf{TM}} = \{\langle M , x \rangle | \text{ M is a TM that rejects input} \} $$
	
	Show directly (i.e., without resorting to reduction) that $\textsf{REJECT}_{\textsf{TM}}$ is undecidable.
	
	
	{\em Proof}: 
	
	\question{3}{Reverse on TM}
	
	$$ \textsf{REJECT}_{\textsf{TM}} = \{\langle M , x \rangle | \text{ M is a TM that rejects input} \} $$
	
	Show directly (i.e., without resorting to reduction) that $\textsf{REJECT}_{\textsf{TM}}$ is undecidable.
	
	{\em Proof}: 
		
	\question{4}{Undecidability}
	
	$$ \textsf{REJECT}_{\textsf{TM}} = \{\langle M , x \rangle | \text{ M is a TM that rejects input} \} $$
	
	Show directly (i.e., without resorting to reduction) that $\textsf{REJECT}_{\textsf{TM}}$ is undecidable.
	
	{\em Proof}: 
	
	\question{5}{Total Is No Harder Than Finite}
	
	$$ \textsf{REJECT}_{\textsf{TM}} = \{\langle M , x \rangle | \text{ M is a TM that rejects input} \} $$
	
	Show directly (i.e., without resorting to reduction) that $\textsf{REJECT}_{\textsf{TM}}$ is undecidable.
	
	{\em Proof}: 
	
	
	\question{6}{Finite Is No Harder Than Total}
	
	$$ \textsf{REJECT}_{\textsf{TM}} = \{\langle M , x \rangle | \text{ M is a TM that rejects input} \} $$
	
	Show directly (i.e., without resorting to reduction) that $\textsf{REJECT}_{\textsf{TM}}$ is undecidable.
	
	{\em Proof}: 
	
	
	\question{7}{Extra: Undecidability of Nontrivial Properties}
	
	$$ \textsf{REJECT}_{\textsf{TM}} = \{\langle M , x \rangle | \text{ M is a TM that rejects input} \} $$
	
	Show directly (i.e., without resorting to reduction) that $\textsf{REJECT}_{\textsf{TM}}$ is undecidable.
	
	{\em Proof}: 
	
	
\end{document}