% You should title the file with a .tex extension (hw1.tex, for example)
\documentclass[a4paper, 11pt]{article}
\usepackage{fancyvrb}
\usepackage{verbatim}
\usepackage{amsmath}
\usepackage{amssymb}
\usepackage{fancyhdr}
\usepackage{graphicx}

\usepackage[margin=1in]{geometry}
\usepackage{tikz}
\usetikzlibrary{automata,positioning,arrows}

\usepackage[margin=1in]{geometry}
\usepackage{listings}
\usepackage{color}

\definecolor{dkgreen}{rgb}{0,0.6,0}
\definecolor{gray}{rgb}{0.5,0.5,0.5}
\definecolor{mauve}{rgb}{0.58,0,0.82}

\lstset{frame=tb,
	language=Java,
	aboveskip=3mm,
	belowskip=3mm,
	showstringspaces=false,
	columns=flexible,
	basicstyle={\small\ttfamily},
	numbers=none,
	numberstyle=\tiny\color{gray},
	keywordstyle=\color{blue},
	commentstyle=\color{dkgreen},
	stringstyle=\color{mauve},
	breaklines=true,
	breakatwhitespace=true,
	tabsize=3
}

\newcommand{\question}[2] {\vspace{.25in} \hrule\vspace{0.5em}
	\noindent{\bf #1: #2} \vspace{0.5em}
	\hrule \vspace{.10in}}
\renewcommand{\part}[1] {\vspace{.10in} {\bf (#1)}}

\newcommand{\myname}{Possawat Sanorkam}
\newcommand{\myemail}{possawat2017@hotmail.com}
\newcommand{\myhwnum}{4}

\setlength{\parindent}{0pt}
\setlength{\parskip}{5pt plus 1pt}

\pagestyle{fancyplain}
\lhead{\fancyplain{}{\textbf{HW\myhwnum}}}      % Note the different brackets!
\rhead{\fancyplain{}{\myname\\ \myemail}}
\chead{\fancyplain{}{ICCS310}}

\begin{document}
	
	\medskip                        % Skip a "medium" amount of space
	% (latex determines what medium is)
	% Also try: \bigskip, \littleskip
	
	\thispagestyle{plain}
	\begin{center}                  % Center the following lines
		{\Large ICCS310: Assignment \myhwnum} \\
		\myname \\
		\myemail \\
		\today \\
	\end{center}
	
	\question{1}{Eh? They Have The Same Cardinality?} %don't delete yet:(}
	Prove the following statements using rigorous mathematical reasoning:
	
	\part{1} $|[0,\frac{1}{2})| = |[0,1)|$
	
	{\em Proof}:
	
	
	\part{2} $|[0,1)| = |(-1,1)|$
	
	{\em Proof}: 
	
	\part{3} $|[0,1)| = |\mathbb{R}|$

	{\em Proof}: 	
	
	\question{2}{The Power Set of A } %don't delete yet:(}
	
	\part{1} Prove that $|2^A| = |\{0, 1\}^A|$.
	
	{\em Proof}: 
	
	\part{2} Prove that $|A| < |\{0, 1\}^A|$ and conclude that $|A| < |2^A|$.
	
	{\em Proof}: 
	
	
	\question{3}{Hamming Code}
	
	Consider applying the Hamming coding scheme to send 8 bits of data. This will require 4 parity bits, so an encoded code word in this scheme is 12 bits long.
	
	\part{1} If the data bits are $d_1,d_2,d_3...d_8$ , what is $\beta_2$ in terms of $d_i$’s?
	
	{\em Solution}: 
	
	\part{2} Encode the following 8-bit data: $01101010$.
	
	{\em Solution}: 
	
	\part{3} Assuming that at most a single single bit flip, decide the following codewords (indicate also whether there was any error):
	
	{\em Solution}: 
	
	(i) 010011111000
	
	(ii) 011101010010
	\\\\\\
	
	\question{4}{Same Number of 0s and 1s}
	
	Consider the language $L = \{w \in \{0, 1\}^{*} | \text{ w contains an equal number of 0s and 1s } \}$. Show that L is (Turing) decidable by providing a TM that decides it (a medium-level detail is preferred).
	
	{\em Proof}: 
	
	
	\question{5}{Infinite DFA}
	
	Show that the following language is (Turing) decideable:
	$$ \text{IDFA} = \{\langle M \rangle| \text{ M is a DFA and L(M) is an infinite language  } \}.$$
	
	{\em Proof}: 
	
	\question{6}{Lucky 9}
	
	\part{1} Let $L_1 \subseteq \Sigma^*$ be defined as
	
	$$
	L_1 = \begin{cases} 
	\varnothing & \text{ if $2^{74207281}-1$ is prime} \\
	\{99\} & \text{ if $2^{74207281}-1$ is not prime}
	\end{cases}
	$$
	
	Prove that $L_1$ is (Turing) decidable.
	
	{\em Proof}: 
	
	\part{2} Let $L_2 \subseteq \Sigma^*$ be defined as
	$$
 	w \in L_2 \iff \text{ $w$ appears somewhere (not necessarily consecutively) in the decimal expansion of $\pi$}
	$$
	Prove that $L_2$ is (Turing) decidable.
	
	
	{\em Proof}: 
	
	\question{7}{$\beta$-reduction}
	
	\part{1} 
	
	{\em Solution}: 
	
	\part{2} 
	
	{\em Solution}: 
	
	\question{8}{Fibonacci}
	Using the functions we have developed (e.g., pred, if\_then\_else, mult, add, etc.), write
	down an explicit $ \lambda $-term fib such that
	$\overline{fib}$ $\overline{n} =_\beta \overline{f(n)}$.
	
	{\em Solution}: 
	
	\question{9}{Power Of 2}

	Implement a  $ \lambda $-term for the $pow(n) = 2^n$
	
	{\em Solution}: 
	
	
\end{document}