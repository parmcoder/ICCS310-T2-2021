% You should title the file with a .tex extension (hw1.tex, for example)
\documentclass[a4paper, 11pt]{article}
\usepackage{fancyvrb}
\usepackage{verbatim}
\usepackage{amsmath}
\usepackage{amssymb}
\usepackage{fancyhdr}
\usepackage{graphicx}

\usepackage[margin=1in]{geometry}
\usepackage{tikz}
\usetikzlibrary{automata,positioning,arrows}

\usepackage[margin=1in]{geometry}
\usepackage{listings}
\usepackage{color}

\definecolor{dkgreen}{rgb}{0,0.6,0}
\definecolor{gray}{rgb}{0.5,0.5,0.5}
\definecolor{mauve}{rgb}{0.58,0,0.82}

\lstset{frame=tb,
	language=Java,
	aboveskip=3mm,
	belowskip=3mm,
	showstringspaces=false,
	columns=flexible,
	basicstyle={\small\ttfamily},
	numbers=none,
	numberstyle=\tiny\color{gray},
	keywordstyle=\color{blue},
	commentstyle=\color{dkgreen},
	stringstyle=\color{mauve},
	breaklines=true,
	breakatwhitespace=true,
	tabsize=3
}

\newcommand{\question}[2] {\vspace{.25in} \hrule\vspace{0.5em}
	\noindent{\bf #1: #2} \vspace{0.5em}
	\hrule \vspace{.10in}}
\renewcommand{\part}[1] {\vspace{.10in} {\bf (#1)}}

\newcommand{\myname}{Possawat Sanorkam}
\newcommand{\myemail}{possawat2017@hotmail.com}
\newcommand{\myhwnum}{4}
\newcommand\tab[1][1cm]{\hspace*{#1}}

\setlength{\parindent}{0pt}
\setlength{\parskip}{5pt plus 1pt}

\pagestyle{fancyplain}
\lhead{\fancyplain{}{\textbf{HW\myhwnum}}}      % Note the different brackets!
\rhead{\fancyplain{}{\myname\\ \myemail}}
\chead{\fancyplain{}{ICCS310}}

\begin{document}
	
	\medskip                        % Skip a "medium" amount of space
	% (latex determines what medium is)
	% Also try: \bigskip, \littleskip
	
	\thispagestyle{plain}
	\begin{center}                  % Center the following lines
		{\Large ICCS310: Assignment \myhwnum} \\
		\myname \\
		\myemail \\
		\today \\
	\end{center}
	
	\question{1}{Eh? They Have The Same Cardinality?} %don't delete yet:(}
	Prove the following statements using rigorous mathematical reasoning:
	
	\part{1} $|[0,\frac{1}{2})| = |[0,1)|$
	
	{\em Proof}: We want to show that $|[0,\frac{1}{2})| = |[0,1)|$ by direct proof.  By definition, let A and B be sets. Say A and B have the same cardinality (size), denoted by $|A| = |B|$, if there exists a bijection between them. 
	
	We want to show that there exist a function $f$ such that $f$ is a bijection. Let $f:A \rightarrow B$, $A = [0,\frac{1}{2})$, and $B = [0,1)$. Then, let $f(x) = 2x$. Let $a \in A$ and $a' \in A$. From observation, $f(a)$ is unique for any arbitrary $a$. We have that $ ( \forall a \neq a')[ f(a) \neq f(a')]$, so $f$ is injective. Let $b \in B$. From observation, every $b$ can be obtain from some $f(a)$. We have that $ ( \forall b)( \exists a)[ f(a) = b]$, so $f$ is also surjective. According to the definition we stated before, $f$ is a bijective function. Hence, $|A| = |B|$. Therefore, $|[0,\frac{1}{2})| = |[0,1)|$. $\square$
	
	
	\part{2} $|[0,1)| = |(-1,1)|$
	
	{\em Proof}: We want to show that $|[0,1)| = |(-1,1)|$ by direct proof. By definition, let A and B be sets. Say A and B have the same cardinality (size), denoted by $|A| = |B|$, if there exists a bijection between them. In addition, $|A|=|B|$ if and only if $|A| \leq |B|$ and $|B| \leq |A|$. 
	
	We want to show that there exist functions $f$ that is injective and $g$ that is also injective. Let $f:A \rightarrow B$, $g:B \rightarrow A$, $A = [0,1)$, and $B = (-1,1)$. Then, let $f(x) = x$. Let $a \in A$ and $a' \in A$. From observation, $f(a)$ is unique for any arbitrary $a$. We have that $ ( \forall a \neq a')[ f(a) \neq f(a')]$, so $f$ is injective. 
	Then, let $g(x) = \frac{(x+1)}{2}$.  Let $b \in B$ and $b' \in B$. From observation, $g(b)$ is unique for any arbitrary $b$. We have that $ ( \forall b \neq b')[ f(b) \neq f(b')]$, so $g$ is injective. Hence, $|A| \leq |B|$ and $|B| \leq |A|$ which implies that $|A| = |B|$. Therefore, $|[0,1)| = |(-1,1)|$. $\square$
	
	\part{3} $|[0,1)| = |\mathbb{R}|$

	{\em Proof}: We want to show that $|[0,1)| = |\mathbb{R}|$ by direct proof. Besides, we have that $|[0,1)| = |(-1,1)|$ which means we can show that $|\mathbb{R}| = |(-1,1)|$ instead. Say A and B have the same cardinality (size), denoted by $|A| = |B|$, if there exists a bijection between $|A| = |C|$, we have $|B| = |C|$ also.
	
	We want to show that there exist a function $f$ such that $f$ is a bijection. Let $f: A \rightarrow \mathbb{R}$, and $A = (-1,1)$. Then, let $f(x) = \frac{x}{1-x^2}$. This function is continuous on domain $A$ when $x \neq 1$ and $x \neq -1$. Let $a \in A$ and $a' \in A$. From observation, $f(a)$ is unique for any arbitrary $a$. We have that $ ( \forall a \neq a')[ f(a) \neq f(a')]$, so $f$ is injective. Let $b \in B$. From observation, every $b$ can be obtain from some $f(a)$. The upper bound of $f(x)$ is $\lim_{x \rightarrow 1} f(x) \approx \infty$ and the lower bound of $f(x)$ is $\lim_{x \rightarrow -1} f(x) \approx	 -\infty$. So, we can cover all element in $\mathbb{R}$.
	We have that $ ( \forall b)( \exists a)[ f(a) = b]$, so $f$ is also surjective. According to the definition we stated before, $f$ is a bijective function. So, $|A| = |\mathbb{R}|$ or $|\mathbb{R}| = |(-1,1)|$. Hence, $|\mathbb{R}| = |(-1,1)|$ implies that $|\mathbb{R}| = |[0,1)|$ also. Therefore, $|[0,1)| = |\mathbb{R}|$. $\square$\\\\\\
		
	\question{2}{The Power Set of A } %don't delete yet:(}
	
	\part{1} Prove that $|2^A| = |\{0, 1\}^A|$.
	
	{\em Proof}: %there is a function that is bijective here, recieve a set of A and return a string.
	We want to show that $|2^A| = |\{0, 1\}^A|$ by direct proof. Say K and B have the same cardinality (size), denoted by $|K| = |B|$, if there exists a bijection between them. In addition, $|K|=|B|$ if and only if $|K| \leq |B|$ and $|B| \leq |K|$. 
	
	We want to show that there exist a function $f$ such that $f$ is a bijection. Let $f:K \rightarrow B$, $g_k: K \rightarrow \{1,0\}$, $K = 2^A$, $a \in A, k \in K$ and $B = \{0, 1\}^A$. Then, let 
	$$f(x) =  \text{binary array of length $|A|$ where each bit represents the presence of $a$ in $x$}$$
	Besides, we say that binary array is just a tuple of $g_k(x)$.
	$$f(x) =  (g_k(x) | k \in A)$$
	Also, let 
	$$g_k(x) = \begin{cases} 
	1 & \text{ if $k \in x$} \\
	0 & \text{ if $k \notin x$}
	\end{cases} $$
	
	So, we have the function that map a set $a$ and represents it in a tuple of bits like ($1,0,...$) of length $|A|$.
	Let $i \in K$ and $i' \in K$. From observation, $f(i)$ is unique for any arbitrary $i$. Besides, we can map all the permutation of binary string to each set. We have that $ ( \forall i \neq i')[ f(i) \neq f(i')]$, so $f$ is injective. Let $b \in B$. From observation, every $b$ can be obtain from some $f(a)$. The upper bound of $f(x)$ is $f(A) = (1,1,...,1)$ which is a tuple of 1s with length of $|A|$ and the lower bound of $f(x)$ is $f(\emptyset) = (0,0,...,0)$ which is a tuple of 0s with length of $|A|$. So, we can cover all element in $B$.
	We have that $ ( \forall b)( \exists a)[ f(a) = b]$, so $f$ is also surjective. According to the definition we stated before, $f$ is a bijective function. So, $|A| = |B|$. Therefore,  $|2^A| = |\{0, 1\}^A|$. $\square$.
	
	\part{2} Prove that $|A| < |\{0, 1\}^A|$ and conclude that $|A| < |2^A|$.
	
	{\em Proof}: %the permutation of strings will exponentially grow larger as A increase
	Let A be a nonempty set, though it is potentially countably infinite.
	We want to show that $|\{0, 1\}^A|$ is not countable and then show that $|A| < |\{0, 1\}^A|$.
	
	Assume for the sake of contradiction that $|\{0, 1\}^A|$ is countable, so $|A| \geq |\{0, 1\}^A|$. This means, there exists a
	surjective function $f : A \rightarrow \{0, 1\}^A$. Define the following string $d \in \{0, 1\}^A$ so that for $i = 0, 1, 2, . . . ,$
	
	$$d[i] = 1 - f(i)[i]$$ that is, $d[i]$ is taking the $i$-th bit of the string given by $f(i)$ and negating it. We’ll show that $d$ differs from $f (k)$ for
	every $k \in N$. In particular, they disagree on the k-th position, i.e., $d[k] \neq f(k)[k]$. Hence, $f$ cannot possibly be a
	surjective function from $A \rightarrow \{0, 1\}^A$, so it is a contradiction to our assumption.
	
	Next, we want to show that there exist $g: A \rightarrow \{0, 1\}^A$ such that $g$ is injective. We have that $A \subset 2^A$.  Since  $A \subset 2^A$ , we have that $g: A \rightarrow \{0, 1\}^A$ is
	$$g(x) =  \text{binary array of length $|A|$ where each bit represents the presence of $a$ in $x$}$$
	where $a \in A$.
	
	Besides, we say that binary array is just a tuple of $s_k(x)$, where $s_k: A \rightarrow \{1,0\}$, and $k \in A$ .
	$$f(x) =  (s_k(x) | k \in A)$$
	Also, let 
	$$g_k(x) = \begin{cases} 
	1 & \text{ if $k \neq x$} \\
	0 & \text{ if $k = x$}
	\end{cases} $$
	
	So, we have the function that map an element $a$ and represents it in a tuple of bits like ($1,0,...$) of length $|A|$.
	Let $i \in A$ and $i' \in A$. From observation, $f(i)$ is unique for any arbitrary $i$. Since we only take on element of $A$ into the function, we will always get a tuple with only single bit of $1$, meaning that only one element existed. We have that $ ( \forall i \neq i')[ f(i) \neq f(i')]$, so $f$ is injective. From earlier, we showed that $|A| \ngeq |\{0, 1\}^A|$. Hence,  $|A| < |\{0, 1\}^A|$.
	
	Since we showed that $|2^A| = |\{0, 1\}^A|$, it implies that $|A| < |2^A|$ also.

	Therefore, $|A| < |2^A|$. $\square$
	
	\question{3}{Hamming Code}
	
	Consider applying the Hamming coding scheme to send 8 bits of data. This will require 4 parity bits, so an encoded code word in this scheme is 12 bits long.
	
	\part{1} If the data bits are $d_1,d_2,d_3...d_8$ , what is $\beta_2$ in terms of $d_i$’s?
	
	{\em Solution}: $\beta_2 = p_2 \oplus d_1 \oplus d_3 \oplus d_4 \oplus d_6 \oplus d_7$
	
	In case that $\beta_2$ is a parity bit (I am not sure if you mean $p_2$), $\beta_2 = d_1 \oplus d_3 \oplus d_4 \oplus d_6 \oplus d_7$
	
	\part{2} Encode the following 8-bit data: $01101010$.
	
	{\em Solution}: Hamming Code = $(p_1 p_2 d_1 p_4 d_2 d_3 d_4 p_8 d_5 d_6 d_7 d_8)$
	
	Encode $01101010$ by adding the parity bits as followed
	\begin{eqnarray}
	p_1 = d_1 \oplus d_2 \oplus d_4 \oplus d_5 \oplus d_7 = 0 \oplus 1 \oplus 0 \oplus 1 \oplus 1 &=& 1\\
	p_2 = d_1 \oplus d_3 \oplus d_4 \oplus d_6 \oplus d_7  = 0 \oplus 1 \oplus 0 \oplus 0 \oplus 1 &=& 0\\
	p_4 = d_2 \oplus d_3 \oplus d_4 \oplus d_8 = 1 \oplus 1 \oplus 0 \oplus 0 &=& 0\\
	p_8 = d_5 \oplus d_6 \oplus d_7 \oplus d_8 = 1 \oplus 0 \oplus 1 \oplus 0 &=& 0
	\end{eqnarray}
	
	Therefore, encoded bits are $1 0 0 0 110 0 1010$
	
	\part{3} Assuming that at most a single single bit flip, decide the following codewords (indicate also whether there was any error):
	
	{\em Solution}: 
	Hamming Code = $(p_1 p_2 d_1 p_4 d_2 d_3 d_4 p_8 d_5 d_6 d_7 d_8)$
	
	(i) 010011111000
	
	So, $p_1 = 0, p_2 = 1, p_4 = 0,$and $p_8 = 1$.
	\begin{eqnarray}
	\beta_1 = p_1 \oplus d_1 \oplus d_2 \oplus d_4 \oplus d_5 \oplus d_7 = 0 \oplus 0 \oplus 1 \oplus 1 \oplus 1 \oplus 0 &=& 1\\
	\beta_2 = p_2 \oplus d_1 \oplus d_3 \oplus d_4 \oplus d_6 \oplus d_7 = 1 \oplus 0 \oplus 1 \oplus 1 \oplus 0 \oplus 0 &=& 1\\
	\beta_4 = p_4 \oplus d_2 \oplus d_3 \oplus d_4 \oplus d_8 = 0 \oplus 1 \oplus 1\oplus 1 \oplus 0 &=& 1\\
	\beta_8 = p_8 \oplus d_5 \oplus d_6 \oplus d_7 \oplus d_8 = 1 \oplus 1 \oplus 0 \oplus 0 \oplus 0 &=& 0
	\end{eqnarray}
	
	Error Position is $0111_2 = 7$. Corrected Data is $010011011000$.
	
	(ii) 011101010010
	
	So, $\beta_1 = 0, \beta_2 = 1, \beta_4 = 1,$and $\beta_8 = 1$.
	\begin{eqnarray}
	\beta_1 = p_1 \oplus d_1 \oplus d_2 \oplus d_4 \oplus d_5 \oplus d_7 = 0 \oplus 1 \oplus 0 \oplus 0 \oplus 0 \oplus 1 &=& 0\\
	\beta_2 = p_2 \oplus d_1 \oplus d_3 \oplus d_4 \oplus d_6 \oplus d_7 = 1 \oplus 1 \oplus 1 \oplus 0 \oplus 0 \oplus 1 &=& 0\\
	\beta_4 = p_4 \oplus d_2 \oplus d_3 \oplus d_4 \oplus d_8 = 1 \oplus 0 \oplus 1\oplus 0 \oplus 0 &=& 0\\
	\beta_8 = p_8 \oplus d_5 \oplus d_6 \oplus d_7 \oplus d_8 = 1 \oplus 0 \oplus 0 \oplus 1 \oplus 0 &=& 0
	\end{eqnarray}
	
	Data is already correct.
		
	\question{4}{Same Number of 0s and 1s}
	
	Consider the language $L = \{w \in \{0, 1\}^{*} | \text{ w contains an equal number of 0s and 1s } \}$. Show that L is (Turing) decidable by providing a TM that decides it (a medium-level detail is preferred).
	
	{\em Proof}: %Find the machine! That machine count the difference between 1s and 0s
	We will show that $L$ is Turing decidable. The proof is by construction: we will give a $TM$ $M = (Q, \Sigma, \Gamma, \delta, q_0 , q_{accept} , q_{reject})$ that decides $L$:
	
	On input string $w$:
	
	1. Sweep left to right across the tape, check whether 0 and 1 are both existed, reject if it is not.
	
	2. Return the head to the left-hand end of the tape then scan to the right. If the tape contains only $\texttt{\char32}$, it accepts.
	
	3. If in stage 2 a $0$ is found, then replace the left most $0$ with $\texttt{\char32}$. Keep scanning to the right, if there is no $1$s, it rejects.
	
	4. If in stage 3 a $1$ is found, then replace the left most $1$ with $\texttt{\char32}$. Then, go to stage 2.
	
	5. If in stage 2 a $1$ is found, then replace the left most $1$ with $\texttt{\char32}$. Keep scanning to the right, if there is no $0$s, it rejects.
	
	6. If in stage 5 a $0$ is found, then replace the left most $0$ with $\texttt{\char32}$. Then, go to stage 2.
	
	\question{5}{Infinite DFA}
	
	Show that the following language is (Turing) decideable:
	$$ \text{IDFA} = \{\langle M \rangle| \text{ M is a DFA and L(M) is an infinite language  } \}.$$
	
	{\em Proof}: We want to show that $\text{IDFA}$ is decidable. So, we will construct a TM $T$ that decides $\text{IDFA}$. For all DFAs $M$, we want $T(\langle M \rangle)$ to accepts if $L(M)$ is infinite language, else it will reject. %reading vicky's ans
	
	\question{6}{Lucky 9}
	
	\part{1} Let $L_1 \subseteq \Sigma^*$ be defined as
	
	$$
	L_1 = \begin{cases} 
	\varnothing & \text{ if $2^{74207281}-1$ is prime} \\
	\{99\} & \text{ if $2^{74207281}-1$ is not prime}
	\end{cases}
	$$
	
	Prove that $L_1$ is (Turing) decidable.
	
	{\em Proof}:  %Find the machine! Probably the one that find 
	
	We will show that $L$ is Turing decidable. The proof is by construction: we will give a $TM$ $M = (Q, \Sigma, \Gamma, \delta, q_0 , q_{accept} , q_{reject})$ that decides $L$:
	
	On input string $w$:
	
	1. Perform primality test on $2^{74207281}-1$ to check whether it is a prime number or not.
	
	2. If $2^{74207281}-1$ is prime, it rejects.
	
	3. If $2^{74207281}-1$ is not prime, then check if w is $99$ or not. If it is $99$, then it accepts else it rejects. $\square$
	
	PS. 	
	I have looked up for $2^{74207281}-1$ whether it is a prime or not, and it is a prime. Here is the url, https://www.mersenne.org/primes/?press=M74207281.
	
	\part{2} Let $L_2 \subseteq \Sigma^*$ be defined as
	$$
 	w \in L_2 \iff \text{ $w$ appears somewhere (not necessarily consecutively) in the decimal expansion of $\pi$}
	$$
	Prove that $L_2$ is (Turing) decidable.
	
	
	{\em Proof}: %Find the machine! One that keep moving until the sequence is found
	
	We will show that $L$ is Turing decidable. The proof is by construction: we will give a $TM$ $M = (Q, \Sigma, \Gamma, \delta, q_0 , q_{accept} , q_{reject})$ that decides $L$:
	
	By the characteristic of $\pi$, since it is irrational, it has an infinite number of digits in its decimal representation, and does not settle into an infinitely repeating pattern of digits. This implies that for all $w \in L_2$, $M$ can halt and always accept any $s \in \Sigma^*$ since $\Sigma = {9}$. Besides, $\Sigma^*$ is a set of $9$s and $|s|$ is countably infinite.
	
	On input string $w$:
	
	1. Store input w in the left-most end of the tape and mark a space using $\texttt{\char32}$ to seperate between $w$ and $\pi$ value.
	
	2. Repeat the following stage until all stored characters are $\texttt{\char32}$. When all stored characters are $\texttt{\char32}$, it accepts.
	
	\tab	3. Keep listing the value of $\pi$ to the right until $9$ is found.
	
	\tab	4. Move the head back to the left-most end $9$ in stored $w$ and mark it $\texttt{\char32}$ $\square$
	
	
	\question{7}{$\beta$-reduction}
	
	\part{1} $(\lambda z.z)(\lambda z.zz)(\lambda z.z y)$
	
	{\em Solution}:
	
	\begin{eqnarray}
	 (\lambda z.z)(\lambda z.zz)(\lambda z.z y) &\rightarrow_1&  (\lambda z.zz)(\lambda z.z y)\\
	  &\rightarrow_1&  (\lambda z.z y)(\lambda z.z y)\\	  
	  &\rightarrow_1&  (\lambda z.z y)y\\
	  &\rightarrow_1&  yy
	\end{eqnarray}
	
	\part{2} $(((\lambda x.\lambda y.(x y))(\lambda y.y))w)$
	
	{\em Solution}: \begin{eqnarray}
	((  (\lambda x.\lambda y.(x y)) (\lambda y.y) )w) 
	&\rightarrow_1&  ((  (\lambda x.\lambda y.(x y)) (\lambda y'.y') )w) \\
	&\rightarrow_1&  (  \lambda y.((\lambda y'.y')y)w) \\
	&\rightarrow_1&  (  \lambda y.(y)w) \\
	&\rightarrow_1&  w
	\end{eqnarray}
	
	\question{8}{Fibonacci}
	Using the functions we have developed (e.g., $\texttt{pred, if\_then\_else, mult, add}$, etc.), write
	down an explicit $ \lambda $-term fib such that
	$\overline{\texttt{fib} }\;\overline{n} =_\beta \overline{f(n)}$.
	
	%probably recursion i guess?
	{\em Solution}:
	Let $\texttt{fib}(n) = \texttt{plain\_fib}(\texttt{fib}, n)$ for all $n$
	
	Define $T = \lambda zw. z$
	
	We say $\texttt{iszero}$ is a function that return $T$ if the input is zero, else $F$.
	
	Define $\texttt{iszero} := \lambda nxy.n(\lambda z.y)x$
	
	Define $\texttt{or} := \lambda xy.x(T)(y)$
	
	Define $\texttt{plain\_fib} := \lambda fn.\texttt{if\_then\_else}
	(\texttt{or}\; (\texttt{iszero}\;n)(\texttt{iszero}\;(\texttt{pred}\;n)))
	(\overline{1})\\$
	\tab[3.5cm] $(\texttt{add} (f\;\texttt{pred}\; n) (f\;\texttt{pred}\;(\texttt{pred}\;n) ))
	$
	\begin{eqnarray}
	\overline{\texttt{fib} }\;\overline{n} &\rightarrow_*& (\texttt{plain\_fib} \; \overline{\texttt{fib}}) \overline{n}\\
	&\rightarrow_*& \texttt{if\_then\_else}
	(\texttt{or}\; (\texttt{iszero}\;\overline{n})(\texttt{iszero}\;(\texttt{pred}\;\overline{n})))
	\end{eqnarray}
	$\tab[5cm] (\overline{1})(\texttt{add} (\overline{\texttt{fib} }\;\texttt{pred}\; \overline{n}) (\overline{\texttt{fib} }\;\texttt{pred}\;(\texttt{pred}\;\overline{n}) ))$
	
	Therefore,
	$$\overline{\texttt{fib} }\;\overline{n} = \texttt{if\_then\_else}
	(\texttt{or}\; (\texttt{iszero}\;\overline{n})(\texttt{iszero}\;(\texttt{pred}\;\overline{n})))
 (\overline{1})(\texttt{add} (\overline{\texttt{fib} }\;\texttt{pred}\; \overline{n}) (\overline{\texttt{fib} }\;\texttt{pred}\;(\texttt{pred}\;\overline{n}) ))$$
	\question{9}{Power Of 2}

	Implement a  $ \lambda $-term for the $pow(n) = 2^n$
	
	{\em Solution}: $\overline{pow}$ $\overline{n} 
	=  (pow$ $\overline{n}) \overline{2} 
	=  ((\lambda p q . p q) \overline{n}) \overline{2} 
	$
	
	Substitute all church numerals, we get  $\overline{pow}$ $\overline{n} =$
	$$  ((\lambda p q . p q) \lambda f x. f^n x)) \lambda f x. f ( f x) $$
	
	
\end{document}