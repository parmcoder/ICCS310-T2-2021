% You should title the file with a .tex extension (hw1.tex, for example)
\documentclass[a4paper, 11pt]{article}

\usepackage{amsmath}
\usepackage{amssymb}
\usepackage{fancyhdr}

\usepackage[margin=1in]{geometry}

\newcommand{\question}[2] {\vspace{.25in} \hrule\vspace{0.5em}
	\noindent{\bf #1: #2} \vspace{0.5em}
	\hrule \vspace{.10in}}
\renewcommand{\part}[1] {\vspace{.10in} {\bf (#1)}}

\newcommand{\myname}{Possawat Sanorkam}
\newcommand{\myemail}{possawat2017@hotmail.com}
\newcommand{\myhwnum}{1}

\setlength{\parindent}{0pt}
\setlength{\parskip}{5pt plus 1pt}

\pagestyle{fancyplain}
\lhead{\fancyplain{}{\textbf{HW\myhwnum}}}      % Note the different brackets!
\rhead{\fancyplain{}{\myname\\ \myemail}}
\chead{\fancyplain{}{ICCS310}}

\begin{document}
	
	\medskip                        % Skip a "medium" amount of space
	% (latex determines what medium is)
	% Also try: \bigskip, \littleskip
	
	\thispagestyle{plain}
	\begin{center}                  % Center the following lines
		{\Large ICCS310: Assignment \myhwnum} \\
		\myname \\
		\myemail \\
		\today \\
	\end{center}
	
	\question{1}{Review: Something About Sets} %don't delete yet:(}
	
	\part{1} Let $A_1, A_2, A_3$ be any sets from a universe $\mathcal{U}$. 
	 {\em Prove that $ \overline{A_1 \cup A_2 \cup A_3} = \overline{A_1} \cap \overline{A_2} \cap \overline{A_3}$.}\\
	
	{\em Proof}: We want to show that $\overline{A_1 \cup A_2 \cup A_3} \subseteq \overline{A_1} \cap \overline{A_2} \cap \overline{A_3}$
	and $\overline{A_1} \cap \overline{A_2} \cap \overline{A_3} \subseteq \overline{A_1 \cup A_2 \cup A_3} $
	
	Let $A_1, A_2, A_3$ be any three given sets. We'll first prove that
	$\overline{A_1 \cup A_2 \cup A_3} \subseteq \overline{A_1} \cap \overline{A_2} \cap \overline{A_3}$.
	Let $x \in \overline{A_1 \cup A_2 \cup A_3}$. Then, $x \notin A_1 \cup A_2 \cup A_3$ by the definition of complement, so then $x \notin A_1$, $x \notin A_2$ and $x \notin A_3$, by the definition of union. This means that $x \in \overline{A_1}$, $x \in \overline{A_2}$, and $x \in \overline{A_3}$, by the definition of complement. 
	Hence, $x \in \overline{A_1} \cap \overline{A_2} \cap \overline{A_3}$ 
	since x is in $\overline{A_1} $, $ \overline{A_2} $, and $ \overline{A_3}$.

	Also, we will show that $\overline{A_1} \cap \overline{A_2} \cap \overline{A_3} \subseteq \overline{A_1 \cup A_2 \cup A_3} $. Let $y \in \overline{A_1} \cap \overline{A_2} \cap \overline{A_3}$, so y is in $\overline{A_1} $, $ \overline{A_2} $, and $ \overline{A_3}$, by the definition of intersection. 
	This means $y \notin A_1$, $y \notin A_2$, and $y \notin A_3$, by the definition of complement. It follows that $y \notin A_1 \cup A_2 \cup A_3$, and so $y \in \overline{A_1 \cup A_2 \cup A_3}$.
	
	In conclusion, $ \overline{A_1 \cup A_2 \cup A_3} = \overline{A_1} \cap \overline{A_2} \cap \overline{A_3}$.
	
	
	\part{2} Let A and B be any sets from a universe $\mathcal{U}$. 
	{\em Prove that $ \overline{A \cup B} = \overline{A} \cap \overline{B}$.}\\
	
	{\em Proof}: We want to show that $\overline{A \cup B} \subseteq \overline{A} \cap \overline{B}$
	and $\overline{A} \cap \overline{B} \subseteq \overline{A \cup B} $
	
	Let $A \text{ and } B$ be any two given sets. We'll first prove that
	$\overline{A \cup B} \subseteq \overline{A} \cap \overline{B}$.
	Let $x \in \overline{A \cup B}$. Then, $x \notin A \cup B$ by the definition of complement, so then $x \notin A$, and $x \notin B$, by the definition of union. This means that $x \in \overline{A}$, and $x \in \overline{B}$, by the definition of complement. 
	Hence, $x \in \overline{A} \cap \overline{B}$ 
	since x is in $\overline{A} $ and $ \overline{B}$.
	
	Also, we will show that $\overline{A} \cap \overline{B} \subseteq \overline{A \cup B} $. Let $y \in \overline{A} \cap \overline{B}$, so y is in $\overline{A} $ and $ \overline{B}$, by the definition of intersection. 
	This means $y \notin A$ and $y \notin B$, by the definition of complement. It follows that $y \notin A \cup B$, and so $y \in \overline{A \cup B}$.
	
	In conclusion, $ \overline{A \cup B} = \overline{A} \cap \overline{B}$.

	
	\question{2}{Prime and Irrational}
	
	\part{1} Let $p \geq 2$ be a prime and a be a positive integer. Prove that if $p  \text{ divides } a^2$, then $p \text{ divides } a$.\\
	
	{\em Proof}: Using contraposition, we can prove that if $p \text{ does not divides } a$, then $p \text{ does not divides } a^2$ instead.
	
	Let $a$ be any number that cannot be divided by p, so $ a = (p*q) + r$ where $r < p$ and $r,q \in \mathbb{I^+}$. So, we have $a^2 = ((p*q) + r)^2 = (p*q)^2 + 2*(p*q*r) + r^2$. We can see that $r^2$ cannot be divided by $p$. Hence, $p \text{ does not divides } a^2$. 
	
	Therefore, if $p  \text{ divides } a^2$, then $p \text{ divides } a$.\\
	
	\part{2} Prove that if $p$ is any positive prime number, then $\sqrt{p}$ is irrational.\\
	
	{\em Proof}: Assume for the sake of contradiction, let $\sqrt{p}$ be a rational number and p is any positive prime number. Then, $\sqrt{p}=a/b$ where $a$ and $b$ are integers and $b \neq 0$.
	\begin{eqnarray}
	\sqrt{p}&=&a/b\\
	p&=&a^2/b^2\\
	p*b^2&=&a^2
	\end{eqnarray}
	From observation, $p$ divides $a^2$ and that means $p$ divide $a$ (Lemma 2.1).
	Let $a = bq$ for some $q \in \mathbb{I^+}$ and plug it back into equation 3.
	\begin{eqnarray}
	p*b^2&=&p^2q^2\\
	b^2&=&pq^2
	\end{eqnarray}
	From observation, $p$ divides $b^2$ and that means $p$ divide $b$ (Lemma 2.1).
	Since we showed that $	p=a^2/b^2 $ but p is a common factor of $a$ and $b$. We can
	conclude that the equation is false and our assumption is contradicting.
	Hence, $\sqrt{p}$ is irrational.
	Therefore, if $p$ is any positive prime number, then $\sqrt{p}$ is irrational.
	
	\question{3}{Spacing}
	
	Prove that in any set of $n +1$ numbers from $\{1, . . . , 2n\}$, there are always two numbers that are consecutive.
	
	{\em Proof}: Assume for the sake of contradiction that there is no two numbers that are consecutive given any set of $n +1$ numbers from $\{1, . . . , 2n\}$. Let $A$ be the set of $n +1$ numbers from $\{1, . . . , 2n\}$. Using pigeonhole principle, the required size of $A$ is too large to prevent consecutive numbers because the when the size of $A$ is n we could use the set of odd natural number or even natural number where all elements in $A$ is less than $n$. However, the last number to be added will contradict to our assumption that no two numbers are consecutive since there would be no other number to pick, except the consecutive number from $\{1, . . . , 2n\}$. Hence, $A$ is contradicting to our assumption.
	
	Therefore, any set of $n +1$ numbers from $\{1, . . . , 2n\}$, there are always two numbers that are consecutive.
	
	
	\question{4}{Curious Fact about Graphs}
	
	Let $G = (V, E )$ be an undirected graph. Show that G contains two nodes that have equal degrees.
	
	{\em Proof}: Without loss of generality, assume for the sake of contradiction that $G$ is an undirected graph with no loops which has $n$ vertices and every vertices have different degrees. Let D be the set of degrees for each vertex, $D = \{d_1, d_2, d_3, \dots , d_n\}$. Then, $D = \{0,2,...,n-1\}$, since the number of degree is between $0$ to $n-1$.  From observation, we have that there is a vertex with degree $n-1$ and a vertex with degree $0$ which is not possible for any undirected graph. Hence, $D$ contradicted to what we assumed.
	
	Therefore, if $G = (V, E )$ is an undirected graph, then $G$ contains two nodes that have equal degrees.
	
	\question{5}{Basic DFAs}
	Let $\sum = \{a, b, c\}$.
	
	\part{1} The language of strings on $\sum$ whose length is divisible by 5.
	
	{\em Solution}:
	
	\part{2} The language of strings on $\sum$ whose length is either even or divisible by 5 (or both).
	
	{\em Solution}:
	
	\part{3} The language of strings on $\sum$ that has at least one a and contains an even number of bs.
	
	{\em Solution}:
	

	\question{6}{Penultimate}
	\part{1} Prove {\em Theorem}:
	
		{\em Solution}:
	
	
	for any $n \geq 0$, \texttt{Solve\_Hanoi} (n, {From\_Peg}, {To\_Peg}, {Aux\_Peg}) generates exactly $2^n-1$ lines of instruction\\
	
	Predicate : P(x) $\equiv$ for any $x \geq 0$, \texttt{Solve\_Hanoi} (x, ...) generates exactly $2^x-1$ lines of instruction\\
	
	Base case : P(0) $\equiv$ \texttt{Solve\_Hanoi} (n, ...) generates exactly 0 lines of instruction which is true\\
	
	Inductive Steps : Assume that if P(x) is true then P(x+1) is true\\
	P(x)  $\equiv$ \texttt{Solve\_Hanoi} (x, ...) generates exactly $2^x-1$ lines of instruction\\
	P(x+1)  $\equiv$ \texttt{Solve\_Hanoi} (x, ...) generates exactly $2^{x+1}-1$ lines of instruction\\ %do it again
	
	\part{2} Show that any DFA that correctly recognizes$L_2$ must have at least 4 states.
	
	{\em Proof}: Without loss of generality, let us assume for the sake of contradiction that there exist D, a DFA that correctly recognizes $L_2$ using less than 4 states. Let D be a DFA that has 3 states. So, to check whether a string can be recognize requires 4 steps, the first step is to check whether it found 1 yet, the second  step is to check whether the next state is 1 or zero, and there will be two cases to consider, the first case is that the second-to-last symbol of $x$ is 1 and the next symbol is 1 and another case is the second-to-last symbol of $x$ is 1 and the next symbol is 0. We have to start over every time any 0 showed up after we found the first 1. Using pigeonhole principle, the least number of states required is 4 since we showed that 4 different states are needed but the number of states in D is smaller than 4. Hence, D cannot correctly recognize $L_2$ with only 3 states which is contradicting to what our assumption.
	
	Therefore, any DFA that correctly recognizes $L_2$ must have at least 4 states. $\square$
	
	\question{7}{Digit Sum}
	\part{1} Prove {\em Theorem}:
	for any $n \geq 0$, \texttt{Solve\_Hanoi} (n, {From\_Peg}, {To\_Peg}, {Aux\_Peg}) generates exactly $2^n-1$ lines of instruction\\
	
	Predicate : P(x) $\equiv$ for any $x \geq 0$, \texttt{Solve\_Hanoi} (x, ...) generates exactly $2^x-1$ lines of instruction\\
	
	Base case : P(0) $\equiv$ \texttt{Solve\_Hanoi} (n, ...) generates exactly 0 lines of instruction which is true\\
	
	Inductive Steps : Assume that if P(x) is true then P(x+1) is true\\
	P(x)  $\equiv$ \texttt{Solve\_Hanoi} (x, ...) generates exactly $2^x-1$ lines of instruction\\
	P(x+1)  $\equiv$ \texttt{Solve\_Hanoi} (x, ...) generates exactly $2^{x+1}-1$ lines of instruction\\ %do it again
\end{document}