% You should title the file with a .tex extension (hw1.tex, for example)
\documentclass[a4paper, 11pt]{article}

\usepackage{amsmath}
\usepackage{amssymb}
\usepackage{fancyhdr}

\usepackage[margin=1in]{geometry}

\newcommand{\question}[2] {\vspace{.25in} \hrule\vspace{0.5em}
	\noindent{\bf #1: #2} \vspace{0.5em}
	\hrule \vspace{.10in}}
\renewcommand{\part}[1] {\vspace{.10in} {\bf (#1)}}

\newcommand{\myname}{Possawat Sanorkam}
\newcommand{\myemail}{possawat2017@hotmail.com}
\newcommand{\myhwnum}{1}

\setlength{\parindent}{0pt}
\setlength{\parskip}{5pt plus 1pt}

\pagestyle{fancyplain}
\lhead{\fancyplain{}{\textbf{HW\myhwnum}}}      % Note the different brackets!
\rhead{\fancyplain{}{\myname\\ \myemail}}
\chead{\fancyplain{}{ICCS310}}

\begin{document}
	
	\medskip                        % Skip a "medium" amount of space
	% (latex determines what medium is)
	% Also try: \bigskip, \littleskip
	
	\thispagestyle{plain}
	\begin{center}                  % Center the following lines
		{\Large ICCS310: Assignment \myhwnum} \\
		\myname \\
		\myemail \\
		\today \\
	\end{center}
	
	\question{1}{Review: Something About Sets} %don't delete yet:(}
	
	\part{1} Let $A_1, A_2, A_3$ be any sets from a universe $\mathcal{U}$. 
	 {\em Prove that $ \overline{A_1 \cup A_2 \cup A_3} = \overline{A_1} \cap \overline{A_2} \cap \overline{A_3}$.}\\
	
	{\em Proof}: We want to show that $\overline{A_1 \cup A_2 \cup A_3} \subseteq \overline{A_1} \cap \overline{A_2} \cap \overline{A_3}$
	and $\overline{A_1} \cap \overline{A_2} \cap \overline{A_3} \subseteq \overline{A_1 \cup A_2 \cup A_3} $
	
	Let $A_1, A_2, A_3$ be any three given sets. We'll first prove that
	$\overline{A_1 \cup A_2 \cup A_3} \subseteq \overline{A_1} \cap \overline{A_2} \cap \overline{A_3}$.
	Let $x \in \overline{A_1 \cup A_2 \cup A_3}$. Then, $x \notin A_1 \cup A_2 \cup A_3$ by the definition of complement, so then $x \notin A_1$, $x \notin A_2$ and $x \notin A_3$, by the definition of union. This means that $x \in \overline{A_1}$, $x \in \overline{A_2}$, and $x \in \overline{A_3}$, by the definition of complement. 
	Hence, $x \in \overline{A_1} \cap \overline{A_2} \cap \overline{A_3}$ 
	since x is in $\overline{A_1} $, $ \overline{A_2} $, and $ \overline{A_3}$.

	Also, we will show that $\overline{A_1} \cap \overline{A_2} \cap \overline{A_3} \subseteq \overline{A_1 \cup A_2 \cup A_3} $. Let $y \in \overline{A_1} \cap \overline{A_2} \cap \overline{A_3}$, so y is in $\overline{A_1} $, $ \overline{A_2} $, and $ \overline{A_3}$, by the definition of intersection. 
	This means $y \notin A_1$, $y \notin A_2$, and $y \notin A_3$, by the definition of complement. It follows that $y \notin A_1 \cup A_2 \cup A_3$, and so $y \in \overline{A_1 \cup A_2 \cup A_3}$.
	
	In conclusion, $ \overline{A_1 \cup A_2 \cup A_3} = \overline{A_1} \cap \overline{A_2} \cap \overline{A_3}$.
	
	
	\part{2} Let A and B be any sets from a universe $\mathcal{U}$. 
	{\em Prove that $ \overline{A \cup B} = \overline{A} \cap \overline{B}$.}\\
	
	{\em Proof}: We want to show that $\overline{A \cup B} \subseteq \overline{A} \cap \overline{B}$
	and $\overline{A} \cap \overline{B} \subseteq \overline{A \cup B} $
	
	Let $A \text{ and } B$ be any two given sets. We'll first prove that
	$\overline{A \cup B} \subseteq \overline{A} \cap \overline{B}$.
	Let $x \in \overline{A \cup B}$. Then, $x \notin A \cup B$ by the definition of complement, so then $x \notin A$, and $x \notin B$, by the definition of union. This means that $x \in \overline{A}$, and $x \in \overline{B}$, by the definition of complement. 
	Hence, $x \in \overline{A} \cap \overline{B}$ 
	since x is in $\overline{A} $ and $ \overline{B}$.
	
	Also, we will show that $\overline{A} \cap \overline{B} \subseteq \overline{A \cup B} $. Let $y \in \overline{A} \cap \overline{B}$, so y is in $\overline{A} $ and $ \overline{B}$, by the definition of intersection. 
	This means $y \notin A$ and $y \notin B$, by the definition of complement. It follows that $y \notin A \cup B$, and so $y \in \overline{A \cup B}$.
	
	In conclusion, $ \overline{A \cup B} = \overline{A} \cap \overline{B}$.

	
	\question{2}{Prime and Irrational}
	
	\part{1} Let $p \geq 2$ be a prime and a be a positive integer. Prove that if $p  \text{ divides } a^2$, then $p \text{ divides } a$.\\
	
	{\em Proof}: Using contraposition, we can prove that if $p \text{ does not divides } a$, then $p \text{ does not divides } a^2$ instead.
	
	Let $a$ be any number that cannot be divided by p, so $ a = (p*q) + r$ where $r < p$ and $r,q \in \mathbb{I^+}$. So, we have $a^2 = ((p*q) + r)^2 = (p*q)^2 + 2*(p*q*r) + r^2$. We can see that $r^2$ cannot be divided by $p$. Hence, $p \text{ does not divides } a^2$. 
	
	Therefore, if $p  \text{ divides } a^2$, then $p \text{ divides } a$.\\
	
	\part{2} Prove that if $p$ is any positive prime number, then $\sqrt{p}$ is irrational.\\
	
	{\em Proof}: Assume for the sake of contradiction, let $\sqrt{p}$ be a rational number and p is any positive prime number. Then, $\sqrt{p}=a/b$ where $a$ and $b$ are integers and $b \neq 0$.
	\begin{eqnarray}
	\sqrt{p}&=&a/b\\
	p&=&a^2/b^2\\
	p*b^2&=&a^2
	\end{eqnarray}
	From observation, $p$ divides $a^2$ and that means $p$ divide $a$ (Lemma 2.1).
	Let $a = bq$ for some $q \in \mathbb{I^+}$ and plug it back into equation 3.
	\begin{eqnarray}
	p*b^2&=&p^2q^2\\
	b^2&=&pq^2
	\end{eqnarray}
	From observation, $p$ divides $b^2$ and that means $p$ divide $b$ (Lemma 2.1).
	Since we showed that $	p=a^2/b^2 $ but p is a common factor of $a$ and $b$. We can
	conclude that the equation is false and our assumption is contradicting.
	Hence, $\sqrt{p}$ is irrational.
	Therefore, if $p$ is any positive prime number, then $\sqrt{p}$ is irrational.
	
	\question{3}{Spacing}
	
	Prove that in any set of $n +1$ numbers from $\{1, . . . , 2n\}$, there are always two numbers that are consecutive.
	
	{\em Proof}: Without loss of generality, 
	
	
	\question{4}{Curious Fact about Graphs}
	
	Let $G = (V, E )$ be an undirected graph. Show that G contains two nodes that have equal degrees.
	
	{\em Proof}: Without loss of generality, 
	
	\question{5}{Basic DFAs}
	\part{1} Prove {\em Theorem}:
	for any $n \geq 0$, \texttt{Solve\_Hanoi} (n, {From\_Peg}, {To\_Peg}, {Aux\_Peg}) generates exactly $2^n-1$ lines of instruction\\
	
	Predicate : P(x) $\equiv$ for any $x \geq 0$, \texttt{Solve\_Hanoi} (x, ...) generates exactly $2^x-1$ lines of instruction\\
	
	Base case : P(0) $\equiv$ \texttt{Solve\_Hanoi} (n, ...) generates exactly 0 lines of instruction which is true\\
	
	Inductive Steps : Assume that if P(x) is true then P(x+1) is true\\
	P(x)  $\equiv$ \texttt{Solve\_Hanoi} (x, ...) generates exactly $2^x-1$ lines of instruction\\
	P(x+1)  $\equiv$ \texttt{Solve\_Hanoi} (x, ...) generates exactly $2^{x+1}-1$ lines of instruction\\ %do it again
	
	
	To show that this is true in mathematically way, T(x) is the number of line generated from the function using recurrence.
	\begin{eqnarray}
	T(x) &=& 2T(x-1)+1; T(1) = 1; T(0) = 0\\
	T(x) &=& 2^{x-1} + ... + 2 + 1\\
	T(x) &=& 2^x - 1\\
	LHS &=& 2^x - 1;T(x)\\
	RHS &=& 2^x - 1\\
	LHS &\equiv& RHS
	\end{eqnarray}
	
	So, P(x) is true, this time we will prove the P(x+1) by using the equations above.
	\begin{eqnarray}
	T(x+1) &=& 2T(x)+1; T(1) = 1; T(0) = 0\\
	T(x+1) &=& 2^{x} + ... + 2 + 1\\
	T(x+1) &=& 2^{x} + 2^x - 1;
	\end{eqnarray} 
	Using T(x) to solve the equation below\\
	\begin{eqnarray}
	T(x+1) &=& 2^{x+1} - 1\\
	LHS &=& 2^{x+1} - 1;T(x+1)\\
	RHS &=& 2^{x+1} - 1;\\
	LHS &\equiv& RHS
	\end{eqnarray}
	
	From the induction hypothesis, P(x-1) $\implies$ P(x) and P(x) holds for any $x \geq 0$, \texttt{Solve\_Hanoi} (x, ...) generates exactly $2^x-1$ lines of instruction. Q.E.D.\\
	
	\part{3} Prove {\em printRuler}: \\\\
	These are the equations we know from this problem\\
	$f(n)=2f(n-1)+1, f(0)=0$ is number of lines\\
	$g(n)=2g(n-1)+n, g(0)=0$ is number of dashes\\
	$g(n)=a*f(n)+b*n+c$
	\begin{enumerate}
		\item Basically, I just followed the hint
		\begin{eqnarray}
		g(0)&=&a*f(0)+b*0+c\\
		g(0)&=&c\\ 
		g(0)&=&0
		\end{eqnarray} 
		So, c = 0
		
		\item We will find a and b
		\begin{eqnarray}
		g(n)&=&2g(n-1)+n\\
  		a*f(n)+b*n&=&2(a*f(n-1)+b*(n-1))+n\\
  		a*f(n)+b*n&=&2a*f(n-1)+2b*(n-1)+n\\
  		a*f(n)&=&2a*f(n-1)+b*n-2b+n\\
  		a*f(n)-2a*f(n-1)&=&b*n-2b+n
		\end{eqnarray}
		Let's do it side by side
		\begin{eqnarray}  		
		a(f(n)-2f(n-1))&=&b*n+n-2b\\
		a(1)&=&n*(b+1)-2b\\
		a+2b&=&n*(b+1)
		\end{eqnarray}
		\begin{eqnarray}
		a+2b-n*(b+1)&=&0
		\end{eqnarray}
		To find a and b, we know that substitute P and Q = 0 will solve this equation
		\begin{eqnarray}
		P + Qn &=& 0\\
		a+2b&=&P\\
		(b+1)&=&Q\\
		b&=&-1\\
		a&=&2
		\end{eqnarray}
		\begin{eqnarray}
		g(n)&=&a*f(n)+b*n\\
		g(n)&=&2*f(n)-n
		\end{eqnarray}
		
		\item Previously, we got $g(n)=2*f(n)-n$\\
		Also, $f(n) = 2^n-1$. In fact, $g(n)=2^{n+1}-n-2$.
		\item Theorem : $g(n)=2^{n+1}-n-2$ works for all $n\geq0$\\
		Predicate : P(x) $\equiv$ $g(x)=2^{x+1}-x-2$ works for all $x\geq0$\\
		Base case : P(0) $\equiv$ $g(0)=0$ is true\\
		Inductive Steps : Assume that if P(x) is true then P(x+1) is true\\
		P(x)  $\equiv$ $g(x)=2^{x+1}-x-2$\\
		P(x+1)  $\equiv$ $g(x+1)=2^{x+2}-x-3$\\
		Actually, we know that $g(n)=2g(n-1)+n$ has a close form of $g(x)=2^{x+1}-x-2$ according to what we have done on part 2.
		
		So, P(x) is true, this time we will prove the P(x+1) by using the equations above.
		\begin{eqnarray}
		g(x+1)&=&2^{x+2}-x-3\\
		g(x+1)&=&2*2^{x+1}-x-3\\
		g(x+1)&=&2(g(x)+x+2)-x-3\\
		g(x+1)&=&2g(x)+(x+1)\\
		LHS &=& 2^{x+2}-x-3\\
		RHS &=& 2g(x)+(x+1)\\
		LHS &\equiv& RHS
		\end{eqnarray}
		
		From the induction hypothesis, P(x) $\implies$ P(x+1) and P(x) holds for any $x \geq 0$ which will make $g(x)=2^{x+1}-x-2$ true. So, $g(x)=2^{x+1}-x-2$ works. Q.E.D.\\
	\end{enumerate}

	\question{6}{Penultimate}
	\part{1} Prove {\em Theorem}:
	for any $n \geq 0$, \texttt{Solve\_Hanoi} (n, {From\_Peg}, {To\_Peg}, {Aux\_Peg}) generates exactly $2^n-1$ lines of instruction\\
	
	Predicate : P(x) $\equiv$ for any $x \geq 0$, \texttt{Solve\_Hanoi} (x, ...) generates exactly $2^x-1$ lines of instruction\\
	
	Base case : P(0) $\equiv$ \texttt{Solve\_Hanoi} (n, ...) generates exactly 0 lines of instruction which is true\\
	
	Inductive Steps : Assume that if P(x) is true then P(x+1) is true\\
	P(x)  $\equiv$ \texttt{Solve\_Hanoi} (x, ...) generates exactly $2^x-1$ lines of instruction\\
	P(x+1)  $\equiv$ \texttt{Solve\_Hanoi} (x, ...) generates exactly $2^{x+1}-1$ lines of instruction\\ %do it again
	
	\question{7}{Digit Sum}
	\part{1} Prove {\em Theorem}:
	for any $n \geq 0$, \texttt{Solve\_Hanoi} (n, {From\_Peg}, {To\_Peg}, {Aux\_Peg}) generates exactly $2^n-1$ lines of instruction\\
	
	Predicate : P(x) $\equiv$ for any $x \geq 0$, \texttt{Solve\_Hanoi} (x, ...) generates exactly $2^x-1$ lines of instruction\\
	
	Base case : P(0) $\equiv$ \texttt{Solve\_Hanoi} (n, ...) generates exactly 0 lines of instruction which is true\\
	
	Inductive Steps : Assume that if P(x) is true then P(x+1) is true\\
	P(x)  $\equiv$ \texttt{Solve\_Hanoi} (x, ...) generates exactly $2^x-1$ lines of instruction\\
	P(x+1)  $\equiv$ \texttt{Solve\_Hanoi} (x, ...) generates exactly $2^{x+1}-1$ lines of instruction\\ %do it again
\end{document}